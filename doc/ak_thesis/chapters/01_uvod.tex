\chapter{Úvod}
Jedním z cílů oboru počítačové grafiky je vytvářet tak realistický obraz, jak je to jen možné. Takový úkol má ovšem vždy dvě hlediska: jak dobře je známa teorie popisující zobrazovaný jev a jak je možné tuto teorii využít v praxi. Omezení ze strany hardwaru či jiné požadavky (např. časová / paměťová náročnost) pak mohou učinit úlohu velmi těžkou i pro jevy z teoretického hlediska lehce popsatelné. Složitost vizualizace (kromě jiných) silně závisí na složitosti zobrazované scény. 
Zobrazování venkovních scén s velkým množstvím vegetace je považováno za obtížnou úlohu počítačové grafiky, právě kvůli vysoké složitosti takových scén. Věrné real-time renderování každého stébla trávy a každého listu stromu až do nejmenších detailů je na současném hardware nerealizovatelné, pokud ovšem nepřipustíme určitou relaxaci problému. V real-time počítačové grafice je třeba občas rezignovat na dokonalé zobrazení přesně podle příslušné teorie. V takovém případě se požaduje, aby přibližný výsledek byl i přesto velmi podobný přesnému řešení a pozorovatel pokud možno rozdíl nepoznal. Naštěstí existují metody, které zachovávají přijatelnou kvalitu a výrazně snižují nároky na výkon. Jedna z nejvýznamnějších skupin technik vychází z faktu, že detail zobrazeného objektu je limitován svou zobrazenou velikostí na výstupním zařízení. Díky konečnému (a vcelku nízkému) rozlišení obrazovky se zobrazí některé detaily do velikosti pod prahem definovaným Nyquist-Shannonovým vzorkovacím teorémem a mohou proto být vypuštěny. To může značně snížit složitost scény. Další zjednodušení scény lze zavést na základě vlastností lidského vnímání. Některé části scény mají malý vliv na celkový vjem pozorovatele. V rámci takových částí je možné lišit se od exaktního řešení i velmi podstatně aniž by se dojem pozorovatele výrazně zhoršil. Metoda řízení úrovně detailu (Level Of Detail - LOD) využívá právě těchto poznatků.
Se vzrůstajícím výkonem výpočetní techniky vzrůstají i požadavky na kvalitu zobrazování. Nepřekvapí tak, že zobrazení pouhého statického stromu již nevyvolává v uživatelích takový dojem realističnosti, jako vykreslení stromu, který se hýbe vlivem působícího větru. Zohledněním dalšího rozměru problému zobrazování vegetace, kterým je čas, se úloha stane ještě složitější.